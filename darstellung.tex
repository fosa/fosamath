% coding:utf-8
\section{Darstellung von Funktionen}
\subsection{Polares Koordinatensystem}
Eine Funktion kann auch im Polaren Koordinatensystem definiert werden. Dabei wird jeder Punkt durch den Abstand zur Ordinate und den Winkel zur x-Achse definiert. 

\subsection{Umrechnung Kartesisch $\rightarrow$ Polar}
\[ \boxed{r = \sqrt{x^2 + y^2}} \]
\[ \boxed{\varphi = \arctan\left(\frac{y}{x}\right)} \]

\subsection{Umrechnung Polar $\rightarrow$ Kartesisch}
\[ \boxed{x = r \cdot \cos{\varphi}} \]
\[ \boxed{y = r \cdot \sin{\varphi}} \]

\subsection{Parameterdarstellung}
Bei der Paramaterdarstellung wird jeder Punkt durch die x- und die y-Koordinate definiert. 
\[ \boxed{f(t) = \left(\begin{matrix} x(t)\\ y(t) \end{matrix}\right)} \]
