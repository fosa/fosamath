% coding:utf-8

%----------------------------------------
%FOSAMATH, a LaTeX-Code for a mathematical summary for basic analysis
%Copyright (C) 2013, Daniel Winz, Ervin Mazlagic, Adrian Imboden, Philipp Langer

%This program is free software; you can redistribute it and/or
%modify it under the terms of the GNU General Public License
%as published by the Free Software Foundation; either version 2
%of the License, or (at your option) any later version.

%This program is distributed in the hope that it will be useful,
%but WITHOUT ANY WARRANTY; without even the implied warranty of
%MERCHANTABILITY or FITNESS FOR A PARTICULAR PURPOSE.  See the
%GNU General Public License for more details.
%----------------------------------------

% coding:utf-8
\section{Integral}
\subsection{Riemann Integral}
\[ \boxed{A_n = \Delta x \sum_{i=0}^{n-1} f(x_i) \quad 
, \Delta x = \frac{b - a}{n} , x_i = i \cdot  \Delta x , I = [a,b]} \]
Riemann-Summe! \\
Falls diese Summe resp. Reihe konvergent ist, so schreibt man
\[ \boxed{\lim_{n \rightarrow \infty} A_n = \int_{a}^{b} f(x) d x \quad 
\text{Riemann Integral von f über I = [a,b]} } \]

\subsection{Integral für ein beliebiges Polynom}
\[ \boxed{\int x^n = \frac{x^{n + 1}}{n + 1}} \]

\section{Eigenschaften des bestimmten Integrals}

\subsection{Summenregel}
\[ \boxed{\int_a^b (f(x) + g(x)) dx = \int_a^b f(x) dx + \int_a^b g(x) dx} \]

\subsection{Faktorregel}
\[ \boxed{\int_a^b \alpha \cdot f(x) dx = \alpha \int_a^b f(x) dx \quad , 
\forall \alpha \in \mathbb{R}} \]

\subsection{Additivität des Integrals}
\[ \boxed{\int_a^b f(x) dx = \int_a^c f(x) dx + \int_c^b f(x) dx \quad , a 
\leq c \leq b} \]
\\
Es gilt für: 
\[ \boxed{\int_a^b x^n dx = \int_0^b x^n dx - \int_0^a x^n dx 
= \frac{b^{n+1} - a^{n+1}}{n + 1} \quad , n \neq -1} \]
\[ \boxed{n = 0: \quad \int_a^b x^0 dx = \int_a^b 1 dx = b - a} \]
