% coding:utf-8
\documentclass[a5paper,10pt,fleqn]{book}

\usepackage[utf8]{inputenc} % für input utf8
\usepackage[T1]{fontenc} %Schriftcodierung mit UTF-8
\usepackage{textcomp} %Erweiterung von fontenc
\usepackage{lmodern} %Erweiterung des Zeichensatzes

\usepackage{graphics}
\usepackage{graphicx}
\usepackage[ngerman]{babel}
\usepackage{amsmath}
%\usepackage[all]{xy}
%\usepackage[xindy]{glossaries}
\usepackage{makeidx}
\usepackage{pdfpages}
\usepackage{graphicx}
\usepackage{hyperref}
\usepackage{amssymb}

\title{Formelsammlung Mathematik}
\author{Daniel Winz, Ervin Mazlagic, Adrian Imboden}
\date{\today}

\begin{document}

\maketitle

\chapter*{Über diese Arbeit}
Dies ist das Ergebnis einer Zusammenarbeit auf Basis freier Texte erstellt von Studierenden der Fachhochschule Luzern. 

Dieses Schriftstück ist lizenziert unter der GPLv2 und der \TeX-  bzw. \LaTeX- Code ist auf \url{github.com/daniw/fosamath} hinterlegt.


\tableofcontents

\chapter{Rechenregeln}
% coding:utf-8

%----------------------------------------
%FOSAMATH, a LaTeX-Code for a mathematical summary for basic analysis
%Copyright (C) 2013, Daniel Winz, Ervin Mazlagic, Adrian Imboden, Philipp Langer

%This program is free software; you can redistribute it and/or
%modify it under the terms of the GNU General Public License
%as published by the Free Software Foundation; either version 2
%of the License, or (at your option) any later version.

%This program is distributed in the hope that it will be useful,
%but WITHOUT ANY WARRANTY; without even the implied warranty of
%MERCHANTABILITY or FITNESS FOR A PARTICULAR PURPOSE.  See the
%GNU General Public License for more details.
%----------------------------------------

% coding:utf-8
\section{Rechenregeln}
\subsection{Bruchrechnen}
\[ \boxed{\frac{a}{b} \cdot \frac{c}{d} = \frac{a \cdot c}{b \cdot d}} \]
\[ \boxed{n \frac{a}{b} = \frac{n \cdot a}{b}} \]
\[ \boxed{\frac{a}{b} : \frac{c}{d} = \frac{a}{b} \cdot \frac{d}{c} = \frac{a \cdot d}{b \cdot c}} \]

\subsection{Polynomdivision}
Das Vorgehen bei der Polynomdivision ist identisch zur schriftlichen Division. \\\\
Beispiel: 
% \[ \boxed{\begin{array}{rrrlrl}
% &(9x^3 &- 6&x^2 &- 8x&):(3x + 4) = \underline{\underline{3x^2 + 2x}}\\
% -&(9x^3 &- 12&x^2)&&\\
% &&6&x^2 &- 8x&\\
% &&-(6&x^2 &- 8x&)\\
% &&&&0&
% \end{array}} \]

\begin{tabular}{|r@{}r@{}r@{}r@{}l@{}r@{}r@{}l|}
\hline
\rule{0pt}{12pt}&&$(9x^3 $&$- 6$&$x^2 $&$- 8x$&$)$&$:(3x + 4) = \underline{\underline{3x^2 + 2x}}$\\
&$-$&$(9x^3 $&$- 12$&$x^2)$&$\downarrow\,\,$&&\\
\cline{2-5}\rule{0pt}{12pt}&&&$6$&$x^2 $&$- 8x$&&\\
&&&$-(6$&$x^2 $&$- 8x$&$)$&\\
\cline{4-7}\rule{0pt}{12pt}&&&&&$0$&&\\
\hline
\end{tabular}


\subsection{Potenzen}
\[ \boxed{a^n = a \cdot a^{n-1} \quad a^1 = a} \]
\[ \boxed{a^0 = 1 \quad \left(0^0\right)\text{ ist nicht definiert}} \]
\[ \boxed{a^{-1} = \frac{1}{a}} \]
\[ \boxed{a^{-n} = \frac{1}{a^n}} \]

\subsection{Potenzgesetze}
\[ \boxed{a^x \cdot a^y = a^{x+y}} \]
\[ \boxed{\frac{a^x}{a^y} = a^{x-y}} \]
\[ \boxed{(a^x)^y = a^{xy}} \]
\[ \boxed{a^x \cdot b^x = \left(ab\right)^x} \]
\[ \boxed{\frac{a^x}{b^x} = \left(\frac{a}{b}\right)^x} \]
\[ \boxed{a^{\frac{p}{q}} = \sqrt[q]{a^p}} \]

\subsection{Wurzeln}
\[ \boxed{\sqrt[n]{a^m} = a^{\frac{m}{n}} \quad \left(a>0; m, n \in \mathbb{N}; m \geq 1; n \geq 2\right)} \]
\[ \boxed{\sqrt[n]{1}=1 \quad \sqrt[n]{0}=0 \quad \sqrt[n]{a^n}=a \quad \left(a>0\right)} \]

\subsection{Wurzelgesetze}
\[ \boxed{\sqrt[n]{a}\cdot \sqrt[n]{b}=\sqrt[n]{a\cdot b}} \]
\[ \boxed{\frac{\sqrt[n]{a}}{\sqrt[n]{b}}=\sqrt[n]{\frac{a}{b}}} \]
\[ \boxed{\left(\sqrt[n]{a}\right)^k=\sqrt[n]{a^k}} \]
\[ \boxed{\sqrt[nk]{a^{mk}}=\sqrt[n]{a^m}} \]
\[ \boxed{\sqrt[n]{\sqrt[m]{a}}=\sqrt[m]{\sqrt[n]{a}}=\sqrt[mn]{a}} \]

\subsection{Logarithmengesetze}
\[ \boxed{y=\log_ax \Leftrightarrow a^y=x} \]
\[ \boxed{\log_a1=0 \quad \log_aa=1} \]
\[ \boxed{\lg a=\log_{10}a \quad \ln a = \log_ea} \]
\[ \boxed{a^{\log_ax}=x} \]
\[ \boxed{\log_a\left(a^x\right)=x} \]
% \subsubsection{Produkte}
\[ \boxed{\log_b\left(x \cdot y\right) = \log_b\left(x\right) + \log_b\left(y\right)} \]
% \subsubsection{Quotienten}
\[ \boxed{\log_b \left( \frac{x}{y} \right) = \log_b\left(x\right) - \log_b\left(y\right)} \]
% \subsubsection{Summen und Differenzen}
%\[ \boxed{\log_b\left(x \cdot y\right) = \log_b\left(x\right) + \log_b\left(y\right)} \]
\[ \boxed{\log_b\left(x + y\right) = \log_b\left(x\right) + \log_b\left(1 + \frac{y}{x}\right)} \]
% \subsubsection{Potenzen}
\[ \boxed{\log_b\left(x^r\right) = r \cdot \log_b\left(x\right)} \]
\[ \boxed{\log_ax=\frac{\lg x}{\lg a}=\frac{\ln x}{\ln a}} \]

\newpage
\subsection{Trigonometrie}

\begin{figure}[h!]
\centering
\includegraphics[width=0.9\textwidth]{einheitskreis.pdf}
\end{figure}

\newpage
\noindent
$H$: Hypotenuse\\
$A$: Ankathete\\
$G$: Gegenkathete
\[ \boxed{\sin\alpha=\frac{G}{H}} \quad \boxed{\cos\alpha=\frac{A}{H}} \quad \boxed{\tan\alpha=\frac{G}{A}} \quad \boxed{\cot\alpha=\frac{A}{G}} \]
\[ \boxed{\sin x = \sqrt{1-\cos^2x} = \sqrt{\frac{\tan^2x}{1+\tan^2x}}} \]
\[ \boxed{\cos x = \sqrt{1-\sin^2x} = \sqrt{\frac{1}{1+\tan^2x}}} \]
\[ \boxed{\tan x = \frac{\sin x}{\sqrt{1-\sin^2x}} = \frac{\sqrt{1-\cos^2x}}{\cos x} = \frac{\sin x}{\cos x}} \]
\[ \boxed{\sin^2 x + \cos^2 x = 1} \]

\subsection{Spezielle Werte der Winkelfunktionen}
% \begin{tabular}{|l|c|c|c|c|c|}
% \hline              & 0°$ = 0$ & 30°$ = \frac{\pi}{6}$ & 45°$ = \frac{\pi}{4}$ & 60°$ = \frac{\pi}{3}$ & 90°$ = \frac{\pi}{2}$ \\
% \hline $f\left(x\right)=\sin x$ & $0$ & $\frac{1}{2}$ & $\frac{1}{2}\sqrt{2}$ & $\frac{1}{2}\sqrt{3}$ & $1$ \\
% \hline $f\left(x\right)=\cos x$ & $1$ & $\frac{1}{2}\sqrt{3}$ & $\frac{1}{2}\sqrt{2}$ & $\frac{1}{2}$ & $0$ \\
% \hline $f\left(x\right)=\tan x$ & $0$ & $\frac{1}{3}\sqrt{3}$ & $1$ & $\sqrt{3}$ & nicht def. \\
% \hline $f\left(x\right)=\cot x$ & nicht def. & $\sqrt{3}$ & $1$ & $\frac{1}{3}\sqrt{3}$ & $0$ \\
% \hline \end{tabular}

% \begin{tabular}{|l||r|r|r|r|}
% \hline $\varphi$                    &          $\sin(\varphi)$&          $\cos(\varphi)$&          $\tan(\varphi)$&          $\cot(\varphi)$\\
% \hline $0^\circ=0$                  &                      $0$&                      $1$&                      $0$&          nicht definiert\\
% \hline $30^\circ=\frac{\pi}{6}$     &            $\frac{1}{2}$&    $\frac{1}{2}\sqrt{3}$&    $\frac{1}{3}\sqrt{3}$&               $\sqrt{3}$\\
% \hline $45^\circ=\frac{\pi}{4}$     &    $\frac{1}{2}\sqrt{2}$&    $\frac{1}{2}\sqrt{2}$&                      $1$&                      $1$\\
% \hline $60^\circ=\frac{\pi}{3}$     &    $\frac{1}{2}\sqrt{3}$&            $\frac{1}{2}$&               $\sqrt{3}$&    $\frac{1}{3}\sqrt{3}$\\
% \hline $90^\circ=\frac{\pi}{2}$     &                      $1$&                      $0$&          nicht definiert&                      $0$\\
% \hline $120^\circ=\frac{2\pi}{3}$   &    $\frac{1}{2}\sqrt{3}$&           $-\frac{1}{2}$&              $-\sqrt{3}$&   $-\frac{1}{3}\sqrt{3}$\\
% \hline $135^\circ=\frac{3\pi}{4}$   &    $\frac{1}{2}\sqrt{2}$&   $-\frac{1}{2}\sqrt{2}$&                     $-1$&                     $-1$\\
% \hline $150^\circ=\frac{5\pi}{6}$   &            $\frac{1}{2}$&   $-\frac{1}{2}\sqrt{3}$&   $-\frac{1}{3}\sqrt{3}$&              $-\sqrt{3}$\\
% \hline $180^\circ=\pi$              &                      $0$&                     $-1$&                      $0$&          nicht definiert\\
% \hline $210^\circ=\frac{7\pi}{6}$   &           $-\frac{1}{2}$&   $-\frac{1}{2}\sqrt{3}$&    $\frac{1}{3}\sqrt{3}$&               $\sqrt{3}$\\
% \hline $225^\circ=\frac{5\pi}{4}$   &   $-\frac{1}{2}\sqrt{2}$&   $-\frac{1}{2}\sqrt{2}$&                      $1$&                      $1$\\
% \hline $240^\circ=\frac{4\pi}{3}$   &   $-\frac{1}{2}\sqrt{3}$&           $-\frac{1}{2}$&               $\sqrt{3}$&    $\frac{1}{3}\sqrt{3}$\\
% \hline $270^\circ=\frac{3\pi}{2}$   &                     $-1$&                      $0$&          nicht definiert&                      $0$\\
% \hline $300^\circ=\frac{5\pi}{3}$  &   $-\frac{1}{2}\sqrt{3}$&            $\frac{1}{2}$&              $-\sqrt{3}$&   $-\frac{1}{3}\sqrt{3}$\\
% \hline $315^\circ=\frac{7\pi}{4}$   &   $-\frac{1}{2}\sqrt{2}$&    $\frac{1}{2}\sqrt{2}$&                     $-1$&                     $-1$\\
% \hline $330^\circ=\frac{11\pi}{6}$  &           $-\frac{1}{2}$&    $\frac{1}{2}\sqrt{3}$&   $-\frac{1}{3}\sqrt{3}$&              $-\sqrt{3}$\\
% \hline $360^\circ=2 \pi$            &                      $0$&                      $1$&                      $0$&          nicht definiert\\
% \hline\end{tabular}

\[ \begin{array}{|l||r|r|r|r|}
\hline \varphi                    &          \sin(\varphi)&          \cos(\varphi)&          \tan(\varphi)&          \cot(\varphi)\\
\hline 0^\circ=0                  &                      0&                      1&                      0& \text{nicht definiert}\\
\hline 30^\circ=\frac{\pi}{6}     &            \frac{1}{2}&    \frac{1}{2}\sqrt{3}&    \frac{1}{3}\sqrt{3}&               \sqrt{3}\\
\hline 45^\circ=\frac{\pi}{4}     &    \frac{1}{2}\sqrt{2}&    \frac{1}{2}\sqrt{2}&                      1&                      1\\
\hline 60^\circ=\frac{\pi}{3}     &    \frac{1}{2}\sqrt{3}&            \frac{1}{2}&               \sqrt{3}&    \frac{1}{3}\sqrt{3}\\
\hline 90^\circ=\frac{\pi}{2}     &                      1&                      0& \text{nicht definiert}&                      0\\
\hline 120^\circ=\frac{2\pi}{3}   &    \frac{1}{2}\sqrt{3}&           -\frac{1}{2}&              -\sqrt{3}&   -\frac{1}{3}\sqrt{3}\\
\hline 135^\circ=\frac{3\pi}{4}   &    \frac{1}{2}\sqrt{2}&   -\frac{1}{2}\sqrt{2}&                     -1&                     -1\\
\hline 150^\circ=\frac{5\pi}{6}   &            \frac{1}{2}&   -\frac{1}{2}\sqrt{3}&   -\frac{1}{3}\sqrt{3}&              -\sqrt{3}\\
\hline 180^\circ=\pi              &                      0&                     -1&                      0& \text{nicht definiert}\\
\hline 210^\circ=\frac{7\pi}{6}   &           -\frac{1}{2}&   -\frac{1}{2}\sqrt{3}&    \frac{1}{3}\sqrt{3}&               \sqrt{3}\\
\hline 225^\circ=\frac{5\pi}{4}   &   -\frac{1}{2}\sqrt{2}&   -\frac{1}{2}\sqrt{2}&                      1&                      1\\
\hline 240^\circ=\frac{4\pi}{3}   &   -\frac{1}{2}\sqrt{3}&           -\frac{1}{2}&               \sqrt{3}&    \frac{1}{3}\sqrt{3}\\
\hline 270^\circ=\frac{3\pi}{2}   &                     -1&                      0& \text{nicht definiert}&                      0\\
\hline 300^\circ=\frac{5\pi}{3}   &   -\frac{1}{2}\sqrt{3}&            \frac{1}{2}&              -\sqrt{3}&   -\frac{1}{3}\sqrt{3}\\
\hline 315^\circ=\frac{7\pi}{4}   &   -\frac{1}{2}\sqrt{2}&    \frac{1}{2}\sqrt{2}&                     -1&                     -1\\
\hline 330^\circ=\frac{11\pi}{6}  &           -\frac{1}{2}&    \frac{1}{2}\sqrt{3}&   -\frac{1}{3}\sqrt{3}&              -\sqrt{3}\\
\hline 360^\circ=2 \pi            &                      0&                      1&                      0& \text{nicht definiert}\\
\hline\end{array} \]


\subsection{Quadratische Gleichung}
\[ \boxed{f(x) = a \cdot x^2 + b \cdot x + c} \]
\[ \boxed{x_{1,2}=\frac{-b\pm\sqrt{b^2-4ac}}{2a}} \]

\subsection{Binomische Formeln}
Erste Binomische Formel: 
\[ \boxed{(a + b)^2 = a^2 + 2 \cdot a \cdot b + b^2} \]Zweite Binomische Formel: 
\[ \boxed{(a - b)^2 = a^2 - 2 \cdot a \cdot b + b^2} \]Dritte Binomische Formel: 
\[ \boxed{(a + b) \cdot (a - b) = a^2 - b^2} \]

\subsection{Verkettete Funktionen}
\[ \boxed{(f \circ g)(x) := f(g(x))} \]

\subsection{Grenzwerte}
\[ \boxed{\lim\limits_{x \to x_0}(f_1(x) + f_2(x)) = \lim\limits_{x \to x_0}(f_1(x)) + \lim\limits_{x \to x_0}(f_2(x))} \]
\[ \boxed{\lim\limits_{x \to x_0}(f_1(x) - f_2(x)) = \lim\limits_{x \to x_0}(f_1(x)) - \lim\limits_{x \to x_0}(f_2(x))} \]
\[ \boxed{\lim\limits_{x \to x_0}(f_1(x) \cdot f_2(x)) = \lim\limits_{x \to x_0}(f_1(x)) \cdot \lim\limits_{x \to x_0}(f_2(x))} \]
\[ \boxed{\lim\limits_{x \to x_0}\left(\frac{f_1(x)}{f_2(x)}\right) = \frac{\lim\limits_{x \to x_0}(f_1(x))}{\lim\limits_{x \to x_0}(f_2(x))}} \]
\[ \boxed{\lim\limits_{x \to x_0}(c \cdot f_1(x)) = c \cdot \lim\limits_{x \to x_0}(f_1(x))} \]
\[ \boxed{\lim\limits_{x \to x_0}\left(c^{(f_1(x))}\right) = c^{\left(\lim\limits_{x \to x_0}(f_1(x))\right)}} \]
\[ \boxed{\lim\limits_{x \to x_0}\left((f_1(x))^n\right) = \left(\lim\limits_{x \to x_0}(f_1(x))\right)^n} \]
\[ \boxed{\lim\limits_{x \to x_0}\left(\sqrt[n]{(f_1(x))}\right) = \sqrt[n]{\lim\limits_{x \to x_0}(f_1(x))}} \]
\[ \boxed{\lim\limits_{x \to x_0}\left(\log{(f_1(x))}\right) = \log{\lim\limits_{x \to x_0}(f_1(x))}} \]

\subsection{Berechnung von Grenzwerten}
Um Grenzwerte zu ermitteln muss der Ausdruck so angepasst werden, dass eindeutig bestimmt werden kann, was sich daraus ergibt.
Dies wird durch Anwenden der zuvor aufgezeigten Rechenreglen erreicht.\\\\
Bsp.: $ \quad \quad \lim\limits_{n \rightarrow \infty} \left( \dfrac{\alpha \cdot n^2}{n^2-1} \right) $ \\\\
Um den Grenzwert zu ermitteln muss der Ausdruck erweitert werden und zwar so, dass
\begin{itemize}
\item der Term äquivalent bleibt
\item Variabeln des Indizes entfallen
\end{itemize}
Um die geforderten Bedingungen zu erfüllen wird der Term durch den reziproken Wert des Indizes mit der höchsten Potenz (im Zähler!) erweitert. In diesem Fall mit $\frac{1}{n^2}$. 
Eine weitere Regel besagt, dass konstante Faktoren vorangenommen werden können.
Nun sieht es wie folgt aus:\\\\
\indent \indent \indent $ \alpha \lim\limits_{n \rightarrow \infty} 
    \left( \dfrac{\frac{1}{n^2} \cdot (n^2) }{ \frac{1}{n^2} \cdot (n^2 -1) } \right)  $\\\\\\
Vereinfacht man diesen Ausdruck durch elementare Algebra so erhält man: \\\\
\indent \indent \indent $ \alpha \lim\limits_{n \rightarrow \infty} \left( \dfrac{1}{1-
\smash{\underbrace{\frac{1}{n^2}}_0 } \vphantom{\dfrac{1}{n^2}} } \right) $\\\\\\
Der Audruck $\frac{1}{n^2}$ geht für $n \rightarrow \infty$ zu $0$. Daraus ergibt sich das folgende:\\\\\\
\indent \indent \indent $ \alpha \cdot 1 = \alpha $

\ifti
\subsubsection{Berechnung von Grenzwerten mit dem TI-89}\label{subsubsec:limti}
%limit($a_n$, $n$, $\infty)$
\verb?limit(EXP,VAR,POINT[,DIRECTION])?\\\\
\begin{tabular}{@{}lll}
\verb|EXP|	& Ausdruck	& bezeichnet den Term \\
\verb|VAR|	& Variable	& bezeichnet die Variable \\
\verb|POINT|	& Punkt		& bezeichnet den Variablenwert \\
\verb|DIRECTION|& Richtung 	& bezeichnet die Richtung \\
		&		& $\searrow$~~~~von oben: 1 \\
		&		& $\nearrow$~~~~von unten: -1 \\
\end{tabular}\\\\
Bsp.: \verb| limit((x^2-2)/(x-2),x,2,-1)| \\
\indent\indent erzeugt die Ausgabe $\mathtt{ \lim\limits_{x \rightarrow 2^-} (\dfrac{x^2-2}{x-2}) \quad = \quad - \infty } $
\fi
\ifnspire
\subsubsection{Berechnung von Grenzwerten mit dem TI-Nspire}\label{subsubsec:limnspire}
\[ \lim\limits_{\boxed{n}\to\boxed{\infty}^{\boxed{d}}}(\boxed{a_n}) \]\\
\begin{tabular}{@{}llp{6cm}}
$\boxed{a_n}$    & Ausdruck & bezeichnet den Term \\
$\boxed{n}$      & Variable & bezeichnet die Variable \\
$\boxed{\infty}$ & Punkt    & bezeichnet den Variablenwert \\
$\boxed{d}$      & Richtung & bezeichnet die Richtung \\
                 &          & $\searrow$ : von oben: + \\
                 &          & $\nearrow$ : von unten: - \\
		 &          & Wenn keine Richtung gefordert ist, kann dieses Feld leer gelassen werden. 
\end{tabular}
\fi

\subsubsection{L'Hopital}
Die Regel von L'Hopital besagt, dass wenn man einen Grenzwert der Form 
\[ \lim\limits_{x \rightarrow x_0} \frac{f(x)}{g(x)} \Rightarrow \lim\limits_{x \rightarrow x_0} f(x) = \lim\limits_{x \rightarrow x_0} g(x) = 0 \Rightarrow \lim\limits_{x \rightarrow x_0} \frac{0}{0} \] hat, kann der Grenzwert auch über die Ableitungen ermittelt werden, falls der Grenzwert existiert.
Erhält man wieder einen unbestimmten Ausdruck, so kann erneut die Regel von L'Hopital angewendet werden (dies kann man so oft wie nötig wiederholen).
\[ \boxed{ \lim\limits_{x \rightarrow x_0} \frac{f(x)}{g(x)} = \lim\limits_{x \rightarrow x_0} \frac{f'(x)}{g'(x)} \quad} \quad \text{falls Bedingungen erfüllt sind!}\]

\subsection{Fakultät}
\[ \boxed{K! = 1 \cdot 2 \cdot 3\cdot \ldots \cdot K} \qquad \boxed{0! := 1} \]
\[ \boxed{(K + 1)! = 1  \cdot 2 \cdot \ldots \cdot K \cdot (K + 1) = (K + 1) \cdot K!} \]
\[ \boxed{(K - 1)! = 1  \cdot 2 \cdot \ldots \cdot (K - 1) = 1  \cdot 2 \cdot \ldots \cdot (K - 1) \cdot \frac{K}{K} = \frac{K!}{K}} \]
\subsubsection{Einige Fakultäten}
\[ \boxed{\begin{array}{rcr}
\rowcolor{white}  0!&=&1\\
\rowcolor{lgray}  1!&=&1\\
\rowcolor{white}  2!&=&2\\
\rowcolor{lgray}  3!&=&6\\
\rowcolor{white}  4!&=&24\\
\rowcolor{lgray}  5!&=&120\\
\rowcolor{white}  6!&=&720\\
\rowcolor{lgray}  7!&=&5'040\\
\rowcolor{white}  8!&=&40'320\\
\rowcolor{lgray}  9!&=&362'880\\
\rowcolor{white} 10!&=&3'628'800\\
\rowcolor{lgray} 11!&=&39'916'800\\
\rowcolor{white} 12!&=&479'001'600\\
\rowcolor{lgray} 13!&=&6'227'020'800\\
\rowcolor{white} 14!&=&87'178'291'200\\
\rowcolor{lgray} 15!&=&1'307'674'368'000\\
\rowcolor{white} 16!&=&20'922'789'888'000\\
\rowcolor{lgray} 17!&=&355'687'428'096'000\\
\rowcolor{white} 18!&=&6'402'373'705'728'000\\
\rowcolor{lgray} 19!&=&121'645'100'408'832'000\\
\rowcolor{white} 20!&=&2'432'902'008'176'640'000\\
\end{array}}\]


\chapter{Vektorgeometrie}
% coding:utf-8
\section{Vektorgeometrie in der Ebene}

\subsection{Abstand zweier Puntke}
\[ \boxed{ \overline{P_1 P_2} = \sqrt{ (x_2 - x_1)^2 + (y_2 - y_1)^2 } } \]

\subsection{Geradengleichungen}

\subsubsection{Normalform (explizite Form)}
\[ \boxed{ g: y= mx + q }\]
\[ \boxed{ \text{Steigung } m = \frac{y_2 - y_1}{x_2 - x_1} = \frac{\Delta y}{\Delta x}  = tan \varphi } \]

\subsubsection{Koordinatenform (implizite Form)}
\[ \boxed{ g: ax + by + c = 0 } \]

\subsubsection{Achsenabschnittsform}
\[ \boxed{ g: \frac{x}{p} + \frac{y}{q} = 1 } \]

\subsubsection{Hesse'sche Normalform}
\[ \boxed{ g:  \frac{ax + by + c}{\sqrt{ a^2 + b^2 } } = 0 }  \]

\subsubsection{Parameterform}
\[ \boxed{ 
	g: \vec{r} = \vec{r_0} + t \cdot \vec{a}  = 
    \left( 
		\begin{array}{cc} 
	  		x_0 \\ y_0
		\end{array}
	\right)
    + t \cdot 
    \left( 
		\begin{array}{cc} 
			a_x \\ a_y
		\end{array}
    \right)  
   }
\]


\subsection{Normalenvektor}
Der Normalenvektor ist ein Vektor, welcher senkrecht auf einem anderen Vektor bzw. einer Geraden liegt. Hier im Beispiel in welchem $ \vec{n} \bot g(x)$
\[ \boxed{ \vec{n} = 
	\left( 
		\begin{array}{cc} 
	  		n_x \\ n_y
		\end{array}
    \right)
    =
    \left( 
		\begin{array}{cc} 
			a \\ b
		\end{array}
    \right)
    =
    \left( 
		\begin{array}{cc} 
			-a_y \\ a_x
		\end{array}
    \right)
} \]
\noindent
Der Richtungsvektor von $g(x)$ ist 
$  
	\left( 
		\begin{array}{cc} 
	  		a_x \\ a_y
		\end{array}
    \right)
    \Rightarrow 
    \left( 
		\begin{array}{cc} 
	  		-a_y \\ a_x
		\end{array}
    \right)
    = \vec{n}
$.

\subsection{Abstand Punkt zu Gerade}
Für eine Gerade $g: ax + by + c = 0$ und einen Punkt $P_1 (x_1 | y_1)$ gilt:
\[ \boxed{ d = \left| \frac{ax_1 + by_1 + c}{\sqrt{a^2 + b^2}} \right| } \]

\subsection{Schnittwinkel zwischen Geraden}
Für den spitzen Schnittwinkel $\varphi$ zwischen den Geraden 

$g_1: y = m_1x + q_1$ und $g_2: y = m_2x + q_2$ gilt:
\[ \boxed{ tan\varphi = \left| \frac{m_2 - m_1}{1 + m_1 \cdot m_2} \right| } \\ \text{für } \varphi \neq 90^{\circ} \]

\[ \boxed{ g_1 || g_2 \Leftrightarrow m_1 = m_2 \text{und } g_1 \bot g_2 \Leftrightarrow m_2 = - \frac{1}{m_1} } \\ \text{für } m_1 \neq 0 \]

\section{Vektorgeometrie im Raum}

\subsection{Ortsvektor}
Ein Ortsvektor beschreibt den Vektor vom Urspung des Koordinatensystems $O(0|0|0)$ zu einem beliebigen Punkt $P(x|y|z)$.
\[	\boxed{ \vec{r} = \overrightarrow{OP} = x\vec{e_x} + y\vec{e_y} + z\vec{e_z} :=
	\left( 
	  \begin{array}{ccc} 
	    x \\ y \\ z
	  \end{array}
	\right) }
\]
\noindent
Die Vektoren $\vec{e_x},\vec{e_y},\vec{e_z}$ sind die Einheitsvektoren des Koordinatensystems (meist einfach 1 ohne Einheit).
\subsection{Länge eines Ortsvektors (Norm bzw. Betrag)}
\[ \boxed{ ||\vec{r}|| = r = \sqrt{x^2 + y^2 + z^2} } \]

\subsection{Länge eines Vektors (Norm bzw. Betrag)}
\[ \boxed{ ||\vec{a}|| = a = \overrightarrow{P_1P_2} = \sqrt{a_x^2 + a_y^2 + a_z^2} } \]
In dieser Form entspricht $a$ der Raumdiagonale im Quader zu $(a_x|a_y|a_z)$.

\subsection{Vektor aus Anfangs- und Endpunkt}
Möchte man den Vektor $\vec{a}$ von $P_1$ (Anfangspunkt) zu $P_2$ (Endpunkt) haben, so rechnet man Anfang - Ende, bzw. $\vec{P_2} - \vec{P_1}$.
\[  \boxed{
    \vec{a} = \overrightarrow{P_1P_2} = \vec{r_2} - \vec{r_1} =
    \left( 
	  \begin{array}{ccc} 
	    x_2 - x_1 \\ y_2 - y_1 \\ z_2 - z_1
	  \end{array}
	\right)
    }
\]

\subsection{Distanz zweier Punkte}
Um die Distanz von $P_1$ zu $P_2$ zu ermitteln, berechnet man die Norm von $\overrightarrow{P_1P_2}$.
\[ \boxed{
   \overline{P_1P_2} = \sqrt{ (x_2 - x_1)^2 + (y_2 - y_1)^2 + (z_2 + z_1)^2 }
   }
\]

\subsection{Mittelpunkt einer Strecke}
\[ \boxed{ \vec{r_M} = \frac{1}{2} \cdot (\vec{r_1} + \vec{r_2}) } \]
\[ \boxed{ \Rightarrow x_M = \frac{x_1 + x_2}{2} \\ y_m = \frac{y_1 + y_2}{2} \\ z_M = \frac{z_1 + z_2}{2} } \]

\section{Rechenoperationen mit Vektoren}

\subsection{Addition/Subtraktion}
\[ \boxed{ \vec{a}\pm\vec{b} =  
    \left( 
	  \begin{array}{ccc} 
	    a_x \\ a_y \\ a_z
	  \end{array}
	\right)
	\pm
	\left( 
	  \begin{array}{ccc} 
	    b_x \\ b_y \\ b_z
	  \end{array}
	\right)
	=
	\left( 
	  \begin{array}{ccc} 
	    a_x \pm b_x \\ a_y \pm b_y \\ a_z \pm b_z
	  \end{array}
	\right)
} \]

\subsection{Multiplikation mit Skalar}
\[ \boxed{ k \cdot \vec{a} = k \cdot 
\left( 
	  \begin{array}{ccc} 
	    a_x \\ a_y \\ a_z
	  \end{array}
	\right)
	=
	\left( 
	  \begin{array}{ccc} 
	    k \cdot a_x \\ k \cdot a_y \\ k \cdot a_z
	  \end{array}
	\right)
	} \\ \text{für } k \in \mathbb{R}
\]

\subsection{Skalarprodukt}
\[ \boxed{ \vec{a} \cdot \vec{b} = a \cdot b \cdot cos(\varphi) = a_x b_x + a_y b_y + a_z b_z } \]
Der Winkel $\varphi$ ist der Zwischenwinkel von $\vec{a}$ und $\vec{b}$.

\noindent
Für $\vec{a} \neq \vec{0}$, $\vec{b} \neq \vec{0}$ gilt: $\vec{a} \bot \vec{b} \Leftrightarrow \vec{a} \cdot \vec{b} = 0$!

\subsection{Winkel zwischen zwei Vektoren}
\[ \boxed{ cos \varphi = \frac{\vec{a} \cdot \vec{b} }{||\vec{a}|| \cdot ||\vec{b}||} } \]
\[ \boxed{ cos \varphi = \frac{a_x b_x + a_y b_y + a_z b_z}{ \sqrt{a_x^2 + a_y^2 + a_z^2} \sqrt{b_x^2 + b_y^2 + b_z^2} } } \]

\subsection{Vektorprodukt (Kreuzprodukt)}
Mit dem Vektorprodukt erhält man einen Vektor der senkrecht zur Ebene steht, also den Normalenvektor zur Ebene.

\[ \boxed{ \vec{c} = \vec{a} \times \vec{b} = 
\left( 
	  \begin{array}{ccc} 
	    a_x \\ a_y \\ a_z
	  \end{array}
	\right)
	\times
	\left( 
	  \begin{array}{ccc} 
	    b_x \\ b_y \\ b_z
	  \end{array}
	\right)
	=
	\left( 
	  \begin{array}{ccc} 
	    a_y b_z - a_z b_y \\ a_z b_x - a_x b_z \\ a_x b_y - a_y b_x
	  \end{array}
	\right)
} \]
\noindent
Die Fläche die von $\vec{a}$ und $\vec{b}$ aufgespannt wird, entspricht der Norm des Vektorprodukts $c=|\vec{a}\times\vec{b}| = a \cdot b \cdot sin \varphi$ .
Zu Beachten ist, dass $\vec{b} \times \vec{a} = -(\vec{a} \times \vec{b}) $

\subsection{Spatprodukt}
Das Spatprodukt entspricht dem Volumen welches von drei Vektoren aufgespannt wird.

\[ \boxed{ (\vec{a},\vec{b},\vec{c}) = (\vec{a} \times \vec{b}) \cdot \vec{c} = (\vec{b} \times \vec{c}) \cdot \vec{a} = (\vec{c} \times \vec{a}) \cdot \vec{b} } \]


\chapter{Folgen und Reihen}
% coding:utf-8
\section{Folgen}
$a_1$ ist das erste Glied einer Folge\\
$a_n$ ist das $n$-te Glied einer Folge

\subsection{rekursive Darstellung}
Bei der rekursiven Darstellung wird eine Folge durch einen Startwert und eine Abbildungsvorschrift dargestellt. \\
\[ \boxed{ \begin{matrix}
\text{Startwert} & f(1) = a_1 \\
\text{Vorschrift} & F(f(1), f(2), \ldots, f(n)) := f(n + 1) = a_{n + 1}
\end{matrix}} \]

\subsection{arithmetische Folgen}
$d$ ist die Differenz zwischen zwei benachbarten Gliedern\\
\[ \boxed{a_{n+1} - a_n = d} \]
\[ \boxed{a_{n+1} = a_n + d} \]
\[ \boxed{ \begin{matrix} 
a_0 :& a_n =& a_0 + n \cdot d \\
a_1 :& a_n =& a_1 + (n - 1)d 
\end{matrix}}\]

\subsection{geometrische Folgen}
$q$ ist der Quotient von zwei benachbarten Gliedern\\
\[ \boxed{\frac{a_{n+1}}{a_n} = q} \]
\[ \boxed{a_{n+1} = a_n * q} \]
\[ \boxed{a_n = q^{n-1} a_1} \]

\subsection{Konvergenz von Folgen}

% coding:utf-8

%----------------------------------------
%FOSAMATH, a LaTeX-Code for a mathematical summary for basic analysis
%Copyright (C) 2013, Daniel Winz, Ervin Mazlagic, Adrian Imboden, Philipp Langer

%This program is free software; you can redistribute it and/or
%modify it under the terms of the GNU General Public License
%as published by the Free Software Foundation; either version 2
%of the License, or (at your option) any later version.

%This program is distributed in the hope that it will be useful,
%but WITHOUT ANY WARRANTY; without even the implied warranty of
%MERCHANTABILITY or FITNESS FOR A PARTICULAR PURPOSE.  See the
%GNU General Public License for more details.
%----------------------------------------

% coding:utf-8
\section{Reihen}
$S_n$ ist die Summe aller Glieder von $a_1$ bis $a_n$. 
\[ \boxed{S_n = \sum_{k=1}^{n} a_k = a_1 + a_2 + a_3 \cdots + a_n} \]

\subsection{arithmetische Reihe}
\[ \boxed{S_n = \left(n + 1\right)\left(a_0 + d \frac{n}{2}\right) = \left(n + 1\right) \frac{a_0 + a_n}{2}} \]

\subsection{geometrische Reihe}
\[ \boxed{\text{Für } q \neq 1: \quad S_n = a_1 \left(  \frac{q^n - 1}{q - 1} \right) = a_1 \left(  \frac{1 - q^n}{1 - q} \right)} \]

\[ \boxed{\begin{array}{ll}
a_0 :& S_n = a_0 \cdot \dfrac{1-q^{n+1}}{1-q} \\ 
& \\
a_1 :& S_n = a_1 \cdot \dfrac{1-q^n}{1-q}
\end{array}} \]

\[ \boxed{\text{Für } q = 1: \quad S_n = a_1 \left(1+1^1 + 1^2 + \ldots + 1^{n-1}\right) = a_1 n} \]
Unendliche geometrische Reihe
\[ \boxed{\text{Für } |q| < 1: S = \lim_{n \rightarrow \infty} S_n = \frac{a_1}{1 - q}} \] 
%Um das $n$-te Glied einer geometrischen Reihen zu berechnen gilt:\\
%$ q = \frac{a_n}{a_{n-1}} = \frac{a_{n-1}}{a_{n-2}} $ \\
%$ \Rightarrow  a_n = a_{(n-1)} \cdot q = a_{(n-2)} \cdot q^2 = a_{(n - (n-1))} \cdot q^{(n-1)} = a_1 \cdot q^{(n-1)} $
%\[ \boxed{ a_n = a_1 \cdot q^{n-1} } \]
%\[ \boxed{ a_{n+1} = a_1 \cdot q^n } \]
$a_1$ ist hier das erste Element der geometrischen Folge!

\subsection{Konvergenz}

\subsubsection*{Quotientenkriterium}
Um die Konvergenzfrage einer geometrischen Reihe zu klären betrachtet man
\[\boxed{ \lim\limits_{n \rightarrow \infty} \left| \frac{a_{n+1}}{a_n} \right| = \lim\limits_{n \rightarrow \infty} \left| q \right|} \]
Hierbei können drei verschiedene Ergebnisse entstehen:
\[ \boxed{\begin{array}{ll}
|q| < 1 & \text{die Reihe konvergiert} \\
|q| > 1 & \text{die Reihe divergiert} \\
|q| = 1 & \text{keine Aussage möglich}
\end{array}} \]
Dies wird oft als Quotientenkriterium bezeichnet.

\subsubsection*{Wurlzelkrierium}
Eine weitere Methode zur Klärung der Konvergenzfrage liefert das so genannte Wurzelkriterium.
\[\boxed{ \lim\limits_{n \rightarrow \infty} \left( |a_k| \right)^{\frac{1}{n}} = q }\]
Daraus können wie beim Quotientenkriterium drei Fälle eintreten:
\[ \boxed{\begin{array}{ll}
|q| < 1 & \text{die Reihe konvergiert} \\
|q| > 1 & \text{die Reihe divergiert} \\
|q| = 1 & \text{keine Aussage möglich} 
\end{array}} \]


\ifti
\subsection{Reihen mit dem TI89 berechnen}
$\sum$\verb{(Ausdruck,Variable,untere Grenze, obere Grenze){ \\
$\sum$\verb{(EXP,VAR,LOW,HIGH){ \\\\
\begin{tabular}{lll}
EXP  & Ausdruck      & bezeichnet den Term der die Reihe beschreibt \\
VAR  & Variable      & bezeichnet die inkrementierte Variable \\
LOW  & untere Grenze & Anfangspunkt der Inkrementierung \\
HIGH & obere Grenze  & Endpunkt der Inkrementierung \\
\end{tabular}
\fi
\ifnspire
\subsection{Reihen mit dem TI-Nspire berechnen}
\[ \sum_{\boxed{K} ~=~ \boxed{0}}^{\boxed{n}}\boxed{a_n} \]\\
\begin{tabular}{lll}
$\boxed{a_n}$  & Ausdruck      & bezeichnet den Term der die Reihe beschreibt \\
$\boxed{K}$     & Variable      & bezeichnet die inkrementierte Variable \\
$\boxed{0}$    & untere Grenze & Anfangspunkt der Inkrementierung \\
$\boxed{n}$ & obere Grenze  & Endpunkt der Inkrementierung \\
\end{tabular}
\fi


\chapter{Differenzieren}
% coding:utf-8
\section{Ableitungsregeln}

\subsection{Grundoperationen}

\subsubsection{Summenregel}
\[ \boxed{ (f(x) + g(x))' = f'(x) + g'(x) } \]
\[ \text{Wichtig: Ableitung einer konstanten Funktion ist Null! } \]
\[ \Rightarrow (f(x) + c)' = f'(x) \text{ für } c \in R \]

\subsubsection{Faktorregel}
\[ \boxed{ (c \cdot f(x))' = c \cdot f'(x) } \]
\[ \text{Ein konstanter Faktor bleiobt beim Differenzieren (Ableiten) erhalten!} \]

\subsubsection{Produkteregel}
\[ \boxed{ (f(x) \cdot g(x))' = f'(x) \cdot g(x) + f(x) \cdot g'(x) } \]

\subsubsection{Quotientenregel}
\[ \boxed{ \left( \frac{f(x)}{g(x)} \right) = \frac{ f'(x) \cdot g(x) - f(x) \cdot g'(x) }{ g^2(x) } } \\ \text{ gilt falls }g(x) \neq 0 \text{ !} \]

\subsubsection{Kettenregel}
\[ \boxed{ (f(g(x)))' = g'(x) \cdot f'(g(x)) } \]

\newpage

\subsection{Spezielle Regeln}

\subsubsection{Exponenten}
\[ \boxed{ (x^n)' = n\cdot x^{(n-1)} } \]
\[ \boxed{ (e^x)' = e^x } \]
\[ \boxed{ (e^{k\cdot x})' = k \cdot e^x } \]
\[ \boxed{ (a^x)' = ln_a (a^x) } \]

\subsubsection{Logarithmen}
\[ \boxed{ (ln(x))' = \frac{1}{x} } \]  
% \[ \boxed{ (a_{log_x})' = \frac{1}{x \cdot ln(a)} } \]

\subsubsection{Trigonometrie}
\[ \boxed{ (sin(x))' = cos(x) } \]  
\[ \boxed{ (cos(x))' = -sin(x) } \] 
\[ \boxed{ (tan(x))' = \frac{1}{cos^2(x)} } \]  
\[ \boxed{ (cot(x))' = -\frac{1}{sn^2(x)} } \]
% coding:utf-8
\section{Kurvendiskussion}

\subsection{Tangentengleichnung}
\[ \boxed{T(x) = f'(x_0)(x - x_0) + f(x_0)} \]

\subsection{Normale zur Tangente}
\[ \boxed{ T(x) = \frac{-1}{f'(x_0)} \cdot (x-x_0) + f(x_0) } \]

\subsection{Steigen und Fallen}

\[ \boxed{ \begin{array}{lll}
f'(x) \geq 0 \text{ auf } I & \Rightarrow  & f \text{ ist monoton wachsend in $I$} \\
f'(x) \leq 0 \text{ auf } I & \Rightarrow  & f \text{ ist monoton fallend in $I$} \\
f'(x) > 0 \text{ auf } I & \Rightarrow  & f \text{ ist streng monoton wachsend in $I$} \\
f'(x) < 0 \text{ auf } I & \Rightarrow  & f \text{ ist streng monoton fallend in $I$}
\end{array} } \]

\noindent
$I$ entspricht einem Intervall! Dies bedeutet, ist $f'(x)$ über den gesamten Bereich immer $\geq 0$ so ist $f$ monoton wachsend.
Ist $f'(x)$ über den gesamten Bereich $\leq 0$ so ist sie monoton fallend.

\subsection{Krümmungsverhalten}

\[ \boxed{ \begin{matrix}
f''(x) > 0 \text{ auf } I & \Rightarrow  & \text{ Kurve ist konvex } \\
f''(x) < 0 \text{ auf } I & \Rightarrow  & \text{ Kurve ist konkav }
\end{matrix} } \]

\subsection{Extremum}
Ein Extremum ist ein Punkt, zu welchem die Ableitung $0$ ergibt.
Solch ein Extremum kann ein Maximum oder Minimum sein.
Zusätzlich ist zu definieren ob es sich um ein lokales oder globales Extremum handelt.

\[ \boxed{ \begin{matrix}
f'(x_0) = 0 \land f''(x_0) < 0 & \Rightarrow & \text{lokales Maximum in $x_0$} \\
f'(x_0) = 0 \land f''(x_0) > 0 & \Rightarrow & \text{lokales Minimum in $x_0$} 
\end{matrix} } \]

\subsection{Wendepunkt}
Als Wendepukt bezeichnet man jene Stelle, an welcher die Krümmung der Funktion wechselt (konkav zu konvex und umgekehrt).
Im Wendepunkt ist die Steigung jeweils von beiden Seiten aus betrachtet (d.h aus $x_0 > 0$ und $x_0<0$) extremal.

\subsubsection{Notwendiges Kriterium}
\[ \boxed{ \begin{matrix}
f''(x_0) = 0 & \Rightarrow & \text{Wendepunkt in $x_0$}
\end{matrix} } \]

\subsubsection{Hinreichendes Kriterium}
\[ \boxed{ \begin{matrix}
f''(x_0) = 0 \land f'''(x_0) \neq 0 & \Rightarrow & \text{Wendepunkt in $x_0$}
\end{matrix} } \]
Achtung: Ist die dritte Ableitung 0, so kann an dieser Stelle trotzdem ein Wendepunkt sein. 

\subsection{Sattelpunkt}
Ein Sattelpunkt ist ein Wendepunkt mit horizontaler Wendetangente.

\[ \boxed{ \begin{matrix}
f'(x_0) =  f''(x_0) = 0 \land f'''(x_0) \neq 0 & \Rightarrow & \text{Sattelpunkt in $x_0$}
\end{matrix} } \]
% coding:utf-8



%----------------------------------------

%FOSAMATH, a LaTeX-Code for a mathematical summary for basic analysis

%Copyright (C) 2013, Daniel Winz, Ervin Mazlagic, Adrian Imboden, Philipp Langer



%This program is free software; you can redistribute it and/or

%modify it under the terms of the GNU General Public License

%as published by the Free Software Foundation; either version 2

%of the License, or (at your option) any later version.



%This program is distributed in the hope that it will be useful,

%but WITHOUT ANY WARRANTY; without even the implied warranty of

%MERCHANTABILITY or FITNESS FOR A PARTICULAR PURPOSE.  See the

%GNU General Public License for more details.

%----------------------------------------


% coding:utf-8

\section{Krümmung}

Die Krümmung $\kappa$ misst wie stark eine Kurve $\gamma$ gekrümmt ist (vgl. Kreis hat eine konstante Krümmung).
Die Krümmung wird unterschieden in mittlere und momentane Krümmung.
\[ \begin{array}{ll}
	\text{mittlere}  & \dfrac{\Delta \alpha}{\Delta s} \\
	& \\
	\text{momentane} & \dfrac{\delta \alpha}{\delta s} \\
\end{array} \]

\[ \boxed{r(t) = \dfrac{1}{|\kappa(t)|} \quad \quad \text{bzw.} \quad \quad r(x) = \dfrac{1}{|\kappa(x)|} } \]
\[ \boxed{\kappa(x) = \frac{y''(x)}{(1 + y'(x)^2)^{\frac{3}{2}}} }\]

\[\boxed{\begin{array}{lllll} 
	\kappa (x) > 0 & \rightarrow & y''(x) \stackrel{!}{>} 0 & \rightarrow & y \text{ ist konvex} \\
	\kappa (x) < 0 & \rightarrow & y''(x) \stackrel{!}{<} 0 & \rightarrow & y \text{ ist konkav} \\
	\kappa (x) = 0 & \rightarrow & y''(x) = 0		& \xrightarrow[]{Achtung!} & \text{Wendepunkt!}
\end{array}}\]
\section{Scheitelpunkte}
Scheitelpunkte sind Stellen an denen die Krümmung maximal ist. 
Um diese zu berechnen muss die Krümmung $\kappa (x)$ abgeleitet und Null gesetzt werden.
\[ \kappa '(x) \stackrel{!}{=} 0  \]
\[ \boxed{\kappa '(x) = \dfrac{ y'''(x)(1+y'(x)^2)^{\frac{3}{2}} - y''(x) \frac{3}{2}(1+y'(x)^2)^{\frac{1}{2}} y''(x) 2y'(x) }{ (1+y'(x)^2)^3 } \stackrel{!}{=} 0 } \]


\chapter{Intgral}
% coding:utf-8
\section{Integral}
\end{document}

