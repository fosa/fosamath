% coding:utf-8

%----------------------------------------
%FOSAMATH, a LaTeX-Code for a mathematical summary for basic analysis
%Copyright (C) 2013, Daniel Winz, Ervin Mazlagic, Adrian Imboden, Philipp Langer

%This program is free software; you can redistribute it and/or
%modify it under the terms of the GNU General Public License
%as published by the Free Software Foundation; either version 2
%of the License, or (at your option) any later version.

%This program is distributed in the hope that it will be useful,
%but WITHOUT ANY WARRANTY; without even the implied warranty of
%MERCHANTABILITY or FITNESS FOR A PARTICULAR PURPOSE.  See the
%GNU General Public License for more details.
%----------------------------------------

% coding:utf-8
\section{Definition komplexe Zahlen}
Eine komplexe Zahl ist ein Vektor mit einer Komponente im realen Teil und 
einer Komponente im imaginären Teil. 

\section{Darstellung von komplexen Zahlen}
\subsection{Gausssche Zahlenebene}
Komplexe Zahlen können nicht auf dem Zahlenstrahl abgebilder werden. Daher wird 
dafür die gausssche Zahlenebene eingeführt. Das ist ein Koordinatensystem. Auf 
der x-Achse liegt der normale Zahlenstrahl. Auf der y-Achse wird die imaginäre 
Achse gelegt. 
\\
\begin{tikzpicture}[domain=-4:4]
  % Raster
  \draw[very thin,color=gray] (-3.9,-2.9) grid (3.9,2.9);
  % x-Achse
  \draw[->] (-4.2,0) -- (4.2,0) node[right] {$Re$};
  % y-Achse
  \draw[->] (0,-3.2) -- (0,3.2) node[above] {$Im$};
  % Linien, Pfeile
  \draw[-latex] (0,0) -- (3,2) node[right] {$3+2j$};
  % Funktionen
%   \draw[color=red] plot (\x,0.5*\x) node[right] {$f(x) =0.5x$};
%   \draw[color=blue] plot (\x,{sin(\x r)}) node[right] {$f(x) = \sin x$};
%   \draw[color=orange] plot (\x,{0.05*exp(\x)}) node[right] {$f(x) = \frac{1}{20} \mathrm e^x$};
\end{tikzpicture}

\subsection{Polardarstellung}


\subsection{Exponentialform}


\section{Rechenregeln}

