% coding:utf-8
\section{Taylorreihen}
Ein Ausdruck der Form 
\[ \boxed{f(x) = \sum_{n = 0}^{\infty} a_n x^n = a_0 + a_1 x + a_2 x^2 + a_3 x^3 \dots a_n x^n} \]
\[ \boxed{a_k = \frac{f^{(K)}(x_0)}{K!}} \]
nennt man eine Potenzreihe
\subsection{Konvergenzradius}
\[ \boxed{a_k = \frac{f^{(K)}(x_0)}{K!}} \]
\[ \boxed{R = \frac{1}{q} = \lim_{n \rightarrow \infty} \left| \frac{a_n}{a_{n + 1}} \right|} \]
\subsection{Taylorreihe mit Entwicklungspunkt $x_0 = 0$}
\[ \boxed{a_k = \frac{f^{(K)}(0)}{K!}} \]
\subsection{Einfache Funktionen als Taylorreihen}
\[ \boxed{e^x = \sum_{K=0}^{\infty} \frac{x^K}{K!}} \]
\[ \boxed{\sin(x) = \sum_{K=0}^{\infty} \frac{(-1)^K x^{2K+1}}{(2K+1)!}} \]
\[ \boxed{\cos(x) = \sum_{K=0}^{\infty} \frac{(-1)^K x^{2K}}{(2K)!}} \]
\[ \boxed{\cosh(x) = \sum_{K=0}^{\infty} \frac{x^{2K}}{(2K)!}} \]
\[ \boxed{\sinh(x) = \sum_{K=0}^{\infty} \frac{x^{2K+1}}{(2K+1)!}} \]
